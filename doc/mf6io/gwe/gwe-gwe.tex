Input to the Groundwater Energy Transport (GWE-GWE) Exchange is read from the file that has type ``GWE6-GWE6'' in the Simulation Name File.

The list of exchanges entered into the EXCHANGEDATA block must be identical to the list of exchanges entered for the GWF-GWF input file.  One way to ensure that this information is identical is to put this list into an external file and refer to this same list using the OPEN/CLOSE functionality in both this EXCHANGEDATA input block and the EXCHANGEDATA input block in the GWF-GWF input file.

\vspace{5mm}
\subsubsection{Structure of Blocks}
\lstinputlisting[style=blockdefinition]{./mf6ivar/tex/exg-gwegwe-options.dat}
\lstinputlisting[style=blockdefinition]{./mf6ivar/tex/exg-gwegwe-dimensions.dat}
\lstinputlisting[style=blockdefinition]{./mf6ivar/tex/exg-gwegwe-exchangedata.dat}

\vspace{5mm}
\subsubsection{Explanation of Variables}
\begin{description}
\input{./mf6ivar/tex/exg-gwegwe-desc.tex}
\end{description}

\vspace{5mm}
\subsubsection{Example Input File}
\lstinputlisting[style=inputfile]{./mf6ivar/examples/exg-gwegwe-example.dat}

\vspace{5mm}
\subsubsection{Available observation types}
GWE-GWE Exchange observations include the simulated flow for any exchange (\texttt{flow-ja-face}). The data required for each GWE-GWE Exchange observation type is defined in table~\ref{table:gwe-gweobstype}. For \texttt{flow-ja-face} observation types, negative and positive values represent a loss from and gain to the first model specified for this exchange.

\begin{longtable}{p{2cm} p{2.75cm} p{2cm} p{1.25cm} p{7cm}}
\caption{Available GWE-GWE Exchange observation types} \tabularnewline

\hline
\hline
\textbf{Exchange} & \textbf{Observation type} & \textbf{ID} & \textbf{ID2} & \textbf{Description} \\
\hline
\endhead

\hline
\endfoot

GWE-GWE & flow-ja-face & exchange number or boundname & -- & Energy flow between model 1 and model 2 for a specified exchange (which is the consecutive exchange number listed in the EXCHANGEDATA block), or the sum of these exchange flows by boundname if boundname is specified.
\label{table:gwe-gweobstype}
\end{longtable}


\vspace{5mm}
\subsubsection{Example Observation Input File}
\lstinputlisting[style=inputfile]{./mf6ivar/examples/exg-gwegwe-example-obs.dat}

