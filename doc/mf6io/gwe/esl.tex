Input to the Energy Source Loading (ESL) Package is read from the file that has type ``ESL6'' in the Name File.  Any number of ESL Packages can be specified for a single groundwater energy transport model.

\vspace{5mm}
\subsubsection{Structure of Blocks}
\vspace{5mm}

\noindent \textit{FOR EACH SIMULATION}
\lstinputlisting[style=blockdefinition]{./mf6ivar/tex/gwe-esl-options.dat}
\lstinputlisting[style=blockdefinition]{./mf6ivar/tex/gwe-esl-dimensions.dat}
\vspace{5mm}
\noindent \textit{FOR ANY STRESS PERIOD}
\lstinputlisting[style=blockdefinition]{./mf6ivar/tex/gwe-esl-period.dat}
\gwepackageperioddescription

\vspace{5mm}
\subsubsection{Explanation of Variables}
\begin{description}
\input{./mf6ivar/tex/gwe-esl-desc.tex}
\end{description}

\vspace{5mm}
\subsubsection{Example Input File}
\lstinputlisting[style=inputfile]{./mf6ivar/examples/gwe-esl-example.dat}

\vspace{5mm}
\subsubsection{Available observation types}
Energy Source Loading Package observations include the simulated energy source loading rates (\texttt{esl}). The data required for each ESL Package observation type is defined in table~\ref{table:gwe-eslobstype}. The \texttt{esl} observation is equal to the simulated energy source loading rate. Negative and positive values for an observation represent a loss from and gain to the GWE model, respectively.

\begin{longtable}{p{2cm} p{2.75cm} p{2cm} p{1.25cm} p{7cm}}
\caption{Available ESL Package observation types} \tabularnewline

\hline
\hline
\textbf{Stress Package} & \textbf{Observation type} & \textbf{ID} & \textbf{ID2} & \textbf{Description} \\
\hline
\endhead

\hline
\endfoot

ESL & esl & cellid or boundname & -- & Energy source loading rate between the groundwater system and an energy source loading boundary or a group of boundaries.
\label{table:gwe-eslobstype}
\end{longtable}

\vspace{5mm}
\subsubsection{Example Observation Input File}
\lstinputlisting[style=inputfile]{./mf6ivar/examples/gwe-esl-example-obs.dat}
